\begin{figure}
    \centering
    \begin{subfigure}[b]{0.22\textwidth}
        \adjincludegraphics[width=\textwidth,trim={0 0 {.78\width} 0},clip]{figures/fkv-graph-example.png}
        \caption{A vertex cover of size 7.}
    \end{subfigure}
    \hspace{1cm}
    \begin{subfigure}[b]{0.22\textwidth}
        \adjincludegraphics[width=\textwidth,trim={{.22\width} 0 {.56\width} 0},clip]{figures/fkv-graph-example.png}
        \caption{A feedback vertex set of size 6.}
    \end{subfigure}
    \hspace{1cm}
    \begin{subfigure}[b]{0.18\textwidth}
        \adjincludegraphics[width=\textwidth,trim={{.44\width} 0 {.38\width} 0},clip]{figures/fkv-graph-example.png}
        \caption{A linear feedback vertex set of size 5.}
    \end{subfigure}

    \begin{subfigure}[b]{0.18\textwidth}
        \adjincludegraphics[width=\textwidth,trim={{.62\width} 0 {.2\width} 0},clip]{figures/fkv-graph-example.png}
        \caption{An odd cycle transversal of size 8.}
    \end{subfigure}
    \hspace{1cm}
    \begin{subfigure}[b]{0.2\textwidth}
        \adjincludegraphics[width=\textwidth,trim={{.8\width} 0 0 0},clip]{figures/fkv-graph-example.png}
        \caption{A $(4, 3)$-hub of size 9.}
    \end{subfigure}

    

    \caption{Examples of $\F$-modulators, with the $\F$-modulator drawn in red.}
    \label{fig:fkv-graph-example}
\end{figure}