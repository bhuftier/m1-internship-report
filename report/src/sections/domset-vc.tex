\section{Faster algorithm for \prob{DominatingSet}/\vc}
\label{section:domset-vc}

A better algorithm than $\O^\star(3^\vc)$ for $\prob{DominatingSet}/\vc$ was known before this internship. However, we developed our own solution and will present our proof as an example of how we can achieve a better algorithm for such a problem and parameter. First of all, Let us recall the \prob{SetCover} problem:

\begin{problem}
    \problemtitle{SetCover}
    \probleminput{A finite set $U$ (called the universe), a collection of subset $\pazocal{G} = \{S_1, \dots, S_m\}$ such that $S_i \subseteq U$}
    \problemquestion{What is the minimum number of subsets from the collection such that their union covers the entire universe $U$? In other words, what is the smallest $k$ such that $\F \subseteq \pazocal{G}$, $|\F| = k$, and $\bigcup_{X \in \F} = U$?}
\end{problem}

For $(U, \pazocal{G})$ an instance of \prob{SetCover}, and for $A \subseteq U$, we denote by $\Phi(A)$ the smallest set cover of $A$.

\begin{lemma}
    \label{lemma:set-cover}
    Given an instance $(U, \pazocal{G})$ of \prob{SetCover}, there exists an algorithm running in time $\O^\star(2^{|U|})$ which computes $\Phi(A)$ for every $A \subseteq U$.
\end{lemma}

\begin{proof}
    Let us denotes $\phi(A)$ the size of the smallest set cover of $A$. We can compute $\phi(A)$ using the following dynamic programming approach:
    \begin{align*}
        &\phi(\emptyset) = 0\\
        &\phi(A) = \min\{1 + \phi(A \backslash S),\ S \in \pazocal{G} \land S \cap A \neq \emptyset\}
    \end{align*}
    
    To compute $\Phi(A)$, follow the same dynamic programming approach: $\Phi(\emptyset) = \emptyset$ and then track the current construction of the best set cover. We can demonstrate that $\Phi(A)$ is one of the smallest set covers of $A$ by induction:
    \begin{itemize}
        \item $\Phi(\emptyset)$ is obviously $\emptyset$.
        \item Suppose $\F \subseteq \pazocal{G}$ is one of the best set covers of $A$. Choose $S \in \F$ and remove it from $A$. You obtain a strictly smaller set, since $S \cap A \neq \emptyset$, or $\F$ would not have been one of the best set covers of $A$. Observe that $\F \backslash \{S\}$ is a set cover of $A \backslash S$. By induction, $\Phi(A \backslash S)$ is one of the best set covers for $A \backslash S$. Thus, we have $|\F \backslash \{S\}| \geq |\Phi(A \backslash S)|$, leading to $|\F| \geq |\Phi(A \backslash S) \cup \{S\}|$. By minimality of $\F$, $\Phi(A \backslash S) \cup \{S\}$ is one of the smallest set covers of $A$. This set will be checked during the construction of $\Phi(A)$, which concludes that $\Phi(A)$ is one of the smallest set covers of $A$.
    \end{itemize}
    
    Each computation takes time $\poly(|U|, |\pazocal{G}|)$, and there are $2^{|U|}$ different subsets $A$. This results in an overall running time of $\O^\star(2^{|U|})$.
\end{proof}

\section{Faster algorithm for \prob{DominatingSet}/\vc}
\label{section:domset-vc}

A better algorithm than $\O^\star(3^\vc)$ for $\prob{DominatingSet}/\vc$ was known before this internship. However, we developed our own solution and will present our proof as an example of how we can achieve a better algorithm for such a problem and parameter. First of all, Let us recall the \prob{SetCover} problem:

\begin{problem}
    \problemtitle{SetCover}
    \probleminput{A finite set $U$ (called the universe), a collection of subset $\pazocal{G} = \{S_1, \dots, S_m\}$ such that $S_i \subseteq U$}
    \problemquestion{What is the minimum number of subsets from the collection such that their union covers the entire universe $U$? In other words, what is the smallest $k$ such that $\F \subseteq \pazocal{G}$, $|\F| = k$, and $\bigcup_{X \in \F} = U$?}
\end{problem}

For $(U, \pazocal{G})$ an instance of \prob{SetCover}, and for $A \subseteq U$, we denote by $\Phi(A)$ the smallest set cover of $A$.

\begin{lemma}
    \label{lemma:set-cover}
    Given an instance $(U, \pazocal{G})$ of \prob{SetCover}, there exists an algorithm running in time $\O^\star(2^{|U|})$ which computes $\Phi(A)$ for every $A \subseteq U$.
\end{lemma}

\begin{proof}
    Let us denotes $\phi(A)$ the size of the smallest set cover of $A$. We can compute $\phi(A)$ using the following dynamic programming approach:
    \begin{align*}
        &\phi(\emptyset) = 0\\
        &\phi(A) = \min\{1 + \phi(A \backslash S),\ S \in \pazocal{G} \land S \cap A \neq \emptyset\}
    \end{align*}
    
    To compute $\Phi(A)$, follow the same dynamic programming approach: $\Phi(\emptyset) = \emptyset$ and then track the current construction of the best set cover. We can demonstrate that $\Phi(A)$ is one of the smallest set covers of $A$ by induction:
    \begin{itemize}
        \item $\Phi(\emptyset)$ is obviously $\emptyset$.
        \item Suppose $\F \subseteq \pazocal{G}$ is one of the best set covers of $A$. Choose $S \in \F$ and remove it from $A$. You obtain a strictly smaller set, since $S \cap A \neq \emptyset$, or $\F$ would not have been one of the best set covers of $A$. Observe that $\F \backslash \{S\}$ is a set cover of $A \backslash S$. By induction, $\Phi(A \backslash S)$ is one of the best set covers for $A \backslash S$. Thus, we have $|\F \backslash \{S\}| \geq |\Phi(A \backslash S)|$, leading to $|\F| \geq |\Phi(A \backslash S) \cup \{S\}|$. By minimality of $\F$, $\Phi(A \backslash S) \cup \{S\}$ is one of the smallest set covers of $A$. This set will be checked during the construction of $\Phi(A)$, which concludes that $\Phi(A)$ is one of the smallest set covers of $A$.
    \end{itemize}
    
    Each computation takes time $\poly(|U|, |\pazocal{G}|)$, and there are $2^{|U|}$ different subsets $A$. This results in an overall running time of $\O^\star(2^{|U|})$.
\end{proof}

\section{Faster algorithm for \prob{DominatingSet}/\vc}
\label{section:domset-vc}

A better algorithm than $\O^\star(3^\vc)$ for $\prob{DominatingSet}/\vc$ was known before this internship. However, we developed our own solution and will present our proof as an example of how we can achieve a better algorithm for such a problem and parameter. First of all, Let us recall the \prob{SetCover} problem:

\begin{problem}
    \problemtitle{SetCover}
    \probleminput{A finite set $U$ (called the universe), a collection of subset $\pazocal{G} = \{S_1, \dots, S_m\}$ such that $S_i \subseteq U$}
    \problemquestion{What is the minimum number of subsets from the collection such that their union covers the entire universe $U$? In other words, what is the smallest $k$ such that $\F \subseteq \pazocal{G}$, $|\F| = k$, and $\bigcup_{X \in \F} = U$?}
\end{problem}

For $(U, \pazocal{G})$ an instance of \prob{SetCover}, and for $A \subseteq U$, we denote by $\Phi(A)$ the smallest set cover of $A$.

\begin{lemma}
    \label{lemma:set-cover}
    Given an instance $(U, \pazocal{G})$ of \prob{SetCover}, there exists an algorithm running in time $\O^\star(2^{|U|})$ which computes $\Phi(A)$ for every $A \subseteq U$.
\end{lemma}

\begin{proof}
    Let us denotes $\phi(A)$ the size of the smallest set cover of $A$. We can compute $\phi(A)$ using the following dynamic programming approach:
    \begin{align*}
        &\phi(\emptyset) = 0\\
        &\phi(A) = \min\{1 + \phi(A \backslash S),\ S \in \pazocal{G} \land S \cap A \neq \emptyset\}
    \end{align*}
    
    To compute $\Phi(A)$, follow the same dynamic programming approach: $\Phi(\emptyset) = \emptyset$ and then track the current construction of the best set cover. We can demonstrate that $\Phi(A)$ is one of the smallest set covers of $A$ by induction:
    \begin{itemize}
        \item $\Phi(\emptyset)$ is obviously $\emptyset$.
        \item Suppose $\F \subseteq \pazocal{G}$ is one of the best set covers of $A$. Choose $S \in \F$ and remove it from $A$. You obtain a strictly smaller set, since $S \cap A \neq \emptyset$, or $\F$ would not have been one of the best set covers of $A$. Observe that $\F \backslash \{S\}$ is a set cover of $A \backslash S$. By induction, $\Phi(A \backslash S)$ is one of the best set covers for $A \backslash S$. Thus, we have $|\F \backslash \{S\}| \geq |\Phi(A \backslash S)|$, leading to $|\F| \geq |\Phi(A \backslash S) \cup \{S\}|$. By minimality of $\F$, $\Phi(A \backslash S) \cup \{S\}$ is one of the smallest set covers of $A$. This set will be checked during the construction of $\Phi(A)$, which concludes that $\Phi(A)$ is one of the smallest set covers of $A$.
    \end{itemize}
    
    Each computation takes time $\poly(|U|, |\pazocal{G}|)$, and there are $2^{|U|}$ different subsets $A$. This results in an overall running time of $\O^\star(2^{|U|})$.
\end{proof}

\section{Faster algorithm for \prob{DominatingSet}/\vc}
\label{section:domset-vc}

A better algorithm than $\O^\star(3^\vc)$ for $\prob{DominatingSet}/\vc$ was known before this internship. However, we developed our own solution and will present our proof as an example of how we can achieve a better algorithm for such a problem and parameter. First of all, Let us recall the \prob{SetCover} problem:

\begin{problem}
    \problemtitle{SetCover}
    \probleminput{A finite set $U$ (called the universe), a collection of subset $\pazocal{G} = \{S_1, \dots, S_m\}$ such that $S_i \subseteq U$}
    \problemquestion{What is the minimum number of subsets from the collection such that their union covers the entire universe $U$? In other words, what is the smallest $k$ such that $\F \subseteq \pazocal{G}$, $|\F| = k$, and $\bigcup_{X \in \F} = U$?}
\end{problem}

For $(U, \pazocal{G})$ an instance of \prob{SetCover}, and for $A \subseteq U$, we denote by $\Phi(A)$ the smallest set cover of $A$.

\begin{lemma}
    \label{lemma:set-cover}
    Given an instance $(U, \pazocal{G})$ of \prob{SetCover}, there exists an algorithm running in time $\O^\star(2^{|U|})$ which computes $\Phi(A)$ for every $A \subseteq U$.
\end{lemma}

\begin{proof}
    Let us denotes $\phi(A)$ the size of the smallest set cover of $A$. We can compute $\phi(A)$ using the following dynamic programming approach:
    \begin{align*}
        &\phi(\emptyset) = 0\\
        &\phi(A) = \min\{1 + \phi(A \backslash S),\ S \in \pazocal{G} \land S \cap A \neq \emptyset\}
    \end{align*}
    
    To compute $\Phi(A)$, follow the same dynamic programming approach: $\Phi(\emptyset) = \emptyset$ and then track the current construction of the best set cover. We can demonstrate that $\Phi(A)$ is one of the smallest set covers of $A$ by induction:
    \begin{itemize}
        \item $\Phi(\emptyset)$ is obviously $\emptyset$.
        \item Suppose $\F \subseteq \pazocal{G}$ is one of the best set covers of $A$. Choose $S \in \F$ and remove it from $A$. You obtain a strictly smaller set, since $S \cap A \neq \emptyset$, or $\F$ would not have been one of the best set covers of $A$. Observe that $\F \backslash \{S\}$ is a set cover of $A \backslash S$. By induction, $\Phi(A \backslash S)$ is one of the best set covers for $A \backslash S$. Thus, we have $|\F \backslash \{S\}| \geq |\Phi(A \backslash S)|$, leading to $|\F| \geq |\Phi(A \backslash S) \cup \{S\}|$. By minimality of $\F$, $\Phi(A \backslash S) \cup \{S\}$ is one of the smallest set covers of $A$. This set will be checked during the construction of $\Phi(A)$, which concludes that $\Phi(A)$ is one of the smallest set covers of $A$.
    \end{itemize}
    
    Each computation takes time $\poly(|U|, |\pazocal{G}|)$, and there are $2^{|U|}$ different subsets $A$. This results in an overall running time of $\O^\star(2^{|U|})$.
\end{proof}

\input{figures/domset-vc.tex}

\begin{theorem}
    Let $G$ be a graph, and let $X$ be a vertex cover of size $|X| = l$. There exists an algorithm that computes $\prob{DominatingSet}/\vc$ in running time $O^\star(2^l)$.
\end{theorem}

\begin{proof}
    Let $I$ denote the graph outside of $X$: $I = G - X$. Note that $I$ is an independent set (see \reffigure{fig:domset-vc}). We can think of $(X, I)$ as an instance of \prob{SetCover}; the vertices of $X$ represent the universe, and each vertex in $I$ represents a subset of the universe defined by its neighbours in $X$. Thus, by \reflemma{lemma:set-cover}, we can compute $\Phi$ for $(X, I)$ in time $O^\star(2^{|X|})$.

    \medskip
    
    Consider the optimization problem of \prob{DominatingSet}: we want to find the smallest possible set of vertices that dominates $G$. Let us call $\D$ the unkown best dominating set of $G$. Assume we know $A = \D \cap X$, then we can find $\D$ in polynomial time (see \reffigure{fig:domset-vc} for an illustration of the construction of $\D$):
    \begin{itemize}
        \item First, $\D$ needs to dominate every vertex in $I$. Let $B$ be the set of all vertices in $I$ with no neighbours in $A$. It is obvious that $I - B$ is dominated by $A$. However, no vertex in $B$ can be dominated by any vertex in $X$, and since $B$ is an independent set, the only way to dominate $B$ is to take every vertex of $B$ in $\D$.
        \item Finally, $\D$ needs to dominate every vertex in $X$. Let $C$ be the set of all non-yet-dominated vertices in $X$ (i.e., vertices with no neighbours in $A$ and no neighbours in $B$). By hypothesis, we know that $C \cap \D = \emptyset$. Therefore, we need to select vertices from $I$ to dominate $C$. Since $\D$ should be of minimum size, we want the smallest set of vertices from $I$ that dominates $C$, which is exactly $\Phi(C)$. Thus we have $\D = A \uplus B \uplus \Phi(C)$.
    \end{itemize}
    
    \medskip

    We succeed in constructing $\D$ by assuming we knew $A = X \cap \D$. Thus, we can test every subset $A \subseteq X$, apply the previous construction to create a dominating set for each $A$, and select the smallest dominating set. Therefore, the size of $\D$ is given by
    $$|\D| = \min_{A \subseteq X}\{|A| + |B| + \Phi(C)\}$$

    Since there are exactly $2^{|X|}$ subsets of $X$, the total running time of this algorithm is $O^\star(2^{|X|}) = O^\star(2^l)$.
\end{proof}

One flaw of this algorithm is that it requires exponential space to store $\Phi$. An interesting area of future work could be to determine if we can find an algorithm with time complexity $O^\star((3 - \varepsilon)^\vc)$ for $\prob{DominatingSet}/\vc$ that uses only polynomial space.

\begin{theorem}
    Let $G$ be a graph, and let $X$ be a vertex cover of size $|X| = l$. There exists an algorithm that computes $\prob{DominatingSet}/\vc$ in running time $O^\star(2^l)$.
\end{theorem}

\begin{proof}
    Let $I$ denote the graph outside of $X$: $I = G - X$. Note that $I$ is an independent set (see \reffigure{fig:domset-vc}). We can think of $(X, I)$ as an instance of \prob{SetCover}; the vertices of $X$ represent the universe, and each vertex in $I$ represents a subset of the universe defined by its neighbours in $X$. Thus, by \reflemma{lemma:set-cover}, we can compute $\Phi$ for $(X, I)$ in time $O^\star(2^{|X|})$.

    \medskip
    
    Consider the optimization problem of \prob{DominatingSet}: we want to find the smallest possible set of vertices that dominates $G$. Let us call $\D$ the unkown best dominating set of $G$. Assume we know $A = \D \cap X$, then we can find $\D$ in polynomial time (see \reffigure{fig:domset-vc} for an illustration of the construction of $\D$):
    \begin{itemize}
        \item First, $\D$ needs to dominate every vertex in $I$. Let $B$ be the set of all vertices in $I$ with no neighbours in $A$. It is obvious that $I - B$ is dominated by $A$. However, no vertex in $B$ can be dominated by any vertex in $X$, and since $B$ is an independent set, the only way to dominate $B$ is to take every vertex of $B$ in $\D$.
        \item Finally, $\D$ needs to dominate every vertex in $X$. Let $C$ be the set of all non-yet-dominated vertices in $X$ (i.e., vertices with no neighbours in $A$ and no neighbours in $B$). By hypothesis, we know that $C \cap \D = \emptyset$. Therefore, we need to select vertices from $I$ to dominate $C$. Since $\D$ should be of minimum size, we want the smallest set of vertices from $I$ that dominates $C$, which is exactly $\Phi(C)$. Thus we have $\D = A \uplus B \uplus \Phi(C)$.
    \end{itemize}
    
    \medskip

    We succeed in constructing $\D$ by assuming we knew $A = X \cap \D$. Thus, we can test every subset $A \subseteq X$, apply the previous construction to create a dominating set for each $A$, and select the smallest dominating set. Therefore, the size of $\D$ is given by
    $$|\D| = \min_{A \subseteq X}\{|A| + |B| + \Phi(C)\}$$

    Since there are exactly $2^{|X|}$ subsets of $X$, the total running time of this algorithm is $O^\star(2^{|X|}) = O^\star(2^l)$.
\end{proof}

One flaw of this algorithm is that it requires exponential space to store $\Phi$. An interesting area of future work could be to determine if we can find an algorithm with time complexity $O^\star((3 - \varepsilon)^\vc)$ for $\prob{DominatingSet}/\vc$ that uses only polynomial space.

\begin{theorem}
    Let $G$ be a graph, and let $X$ be a vertex cover of size $|X| = l$. There exists an algorithm that computes $\prob{DominatingSet}/\vc$ in running time $O^\star(2^l)$.
\end{theorem}

\begin{proof}
    Let $I$ denote the graph outside of $X$: $I = G - X$. Note that $I$ is an independent set (see \reffigure{fig:domset-vc}). We can think of $(X, I)$ as an instance of \prob{SetCover}; the vertices of $X$ represent the universe, and each vertex in $I$ represents a subset of the universe defined by its neighbours in $X$. Thus, by \reflemma{lemma:set-cover}, we can compute $\Phi$ for $(X, I)$ in time $O^\star(2^{|X|})$.

    \medskip
    
    Consider the optimization problem of \prob{DominatingSet}: we want to find the smallest possible set of vertices that dominates $G$. Let us call $\D$ the unkown best dominating set of $G$. Assume we know $A = \D \cap X$, then we can find $\D$ in polynomial time (see \reffigure{fig:domset-vc} for an illustration of the construction of $\D$):
    \begin{itemize}
        \item First, $\D$ needs to dominate every vertex in $I$. Let $B$ be the set of all vertices in $I$ with no neighbours in $A$. It is obvious that $I - B$ is dominated by $A$. However, no vertex in $B$ can be dominated by any vertex in $X$, and since $B$ is an independent set, the only way to dominate $B$ is to take every vertex of $B$ in $\D$.
        \item Finally, $\D$ needs to dominate every vertex in $X$. Let $C$ be the set of all non-yet-dominated vertices in $X$ (i.e., vertices with no neighbours in $A$ and no neighbours in $B$). By hypothesis, we know that $C \cap \D = \emptyset$. Therefore, we need to select vertices from $I$ to dominate $C$. Since $\D$ should be of minimum size, we want the smallest set of vertices from $I$ that dominates $C$, which is exactly $\Phi(C)$. Thus we have $\D = A \uplus B \uplus \Phi(C)$.
    \end{itemize}
    
    \medskip

    We succeed in constructing $\D$ by assuming we knew $A = X \cap \D$. Thus, we can test every subset $A \subseteq X$, apply the previous construction to create a dominating set for each $A$, and select the smallest dominating set. Therefore, the size of $\D$ is given by
    $$|\D| = \min_{A \subseteq X}\{|A| + |B| + \Phi(C)\}$$

    Since there are exactly $2^{|X|}$ subsets of $X$, the total running time of this algorithm is $O^\star(2^{|X|}) = O^\star(2^l)$.
\end{proof}

One flaw of this algorithm is that it requires exponential space to store $\Phi$. An interesting area of future work could be to determine if we can find an algorithm with time complexity $O^\star((3 - \varepsilon)^\vc)$ for $\prob{DominatingSet}/\vc$ that uses only polynomial space.

\begin{theorem}
    Let $G$ be a graph, and let $X$ be a vertex cover of size $|X| = l$. There exists an algorithm that computes $\prob{DominatingSet}/\vc$ in running time $O^\star(2^l)$.
\end{theorem}

\begin{proof}
    Let $I$ denote the graph outside of $X$: $I = G - X$. Note that $I$ is an independent set (see \reffigure{fig:domset-vc}). We can think of $(X, I)$ as an instance of \prob{SetCover}; the vertices of $X$ represent the universe, and each vertex in $I$ represents a subset of the universe defined by its neighbours in $X$. Thus, by \reflemma{lemma:set-cover}, we can compute $\Phi$ for $(X, I)$ in time $O^\star(2^{|X|})$.

    \medskip
    
    Consider the optimization problem of \prob{DominatingSet}: we want to find the smallest possible set of vertices that dominates $G$. Let us call $\D$ the unkown best dominating set of $G$. Assume we know $A = \D \cap X$, then we can find $\D$ in polynomial time (see \reffigure{fig:domset-vc} for an illustration of the construction of $\D$):
    \begin{itemize}
        \item First, $\D$ needs to dominate every vertex in $I$. Let $B$ be the set of all vertices in $I$ with no neighbours in $A$. It is obvious that $I - B$ is dominated by $A$. However, no vertex in $B$ can be dominated by any vertex in $X$, and since $B$ is an independent set, the only way to dominate $B$ is to take every vertex of $B$ in $\D$.
        \item Finally, $\D$ needs to dominate every vertex in $X$. Let $C$ be the set of all non-yet-dominated vertices in $X$ (i.e., vertices with no neighbours in $A$ and no neighbours in $B$). By hypothesis, we know that $C \cap \D = \emptyset$. Therefore, we need to select vertices from $I$ to dominate $C$. Since $\D$ should be of minimum size, we want the smallest set of vertices from $I$ that dominates $C$, which is exactly $\Phi(C)$. Thus we have $\D = A \uplus B \uplus \Phi(C)$.
    \end{itemize}
    
    \medskip

    We succeed in constructing $\D$ by assuming we knew $A = X \cap \D$. Thus, we can test every subset $A \subseteq X$, apply the previous construction to create a dominating set for each $A$, and select the smallest dominating set. Therefore, the size of $\D$ is given by
    $$|\D| = \min_{A \subseteq X}\{|A| + |B| + \Phi(C)\}$$

    Since there are exactly $2^{|X|}$ subsets of $X$, the total running time of this algorithm is $O^\star(2^{|X|}) = O^\star(2^l)$.
\end{proof}

One flaw of this algorithm is that it requires exponential space to store $\Phi$. An interesting area of future work could be to determine if we can find an algorithm with time complexity $O^\star((3 - \varepsilon)^\vc)$ for $\prob{DominatingSet}/\vc$ that uses only polynomial space.