\section{Introduction}
\label{section:introduction}

$\NP$-hard problems are widely recognized as unsolvable in polynomial time unless $\P = \NP$. However, not only do most scientists believe that $\P \neq \NP$, but conjectures such as the Exponential Time Hypothesis (ETH) \cite{impagliazzo2001complexity} offer little hope for discovering even subexponential time algorithms for some of these $\NP$-hard problems. This is one of the main reasons why we study parameterized complexity.

% (maybe remove this part
% Parameterized complexity aims to classify computational problems according to their inherent difficulty with respect to multiple input parameters. The complexity of a problem is then evaluated as a function of these parameters.
% )
Formally, a \textit{parameterized problem} is defined as a language $L \subseteq \Sigma^* \times \N$.
% A parameterized problem $L$ is \textit{fixed-parameter tractable} (\CC{FPT}) if the question "$(x, k) \in L$ ?" can be decided in time $f(k)\cdot |x|^{\O(1)}$, where $f$ is an arbitrary function depending only on $k$.
A \textit{parameterized class} $\CC{C}$ is described by a function $f_{\CC{C}} : \N \times \N \mapsto \N$. We say that $L \in \CC{C}$ if there exists an algorithm that decides whether $(x, k) \in L$ in time $f_{\CC{C}}(|x|, k)$. Of course, we would prefer to find functions that are not exponential in $n = |x|$. Thus, we will focus on the class of parameterized problems that can be solved in $f(k)\cdot n^{\O(1)}$ time, for any function $f$. This class is called \textit{Fixed Parameter Tractable} (\CC{FPT}).

\medskip

But let us dive into a concrete problem to fully understand what we are talking about. Let us recall the \prob{VertexCover} problem:

\begin{problem}
    \problemtitle{VertexCover}
    \probleminput{A graph $G = (V, E)$, an integer $k$}
    \problemquestion{Does there exist a subset of vertices $X\subseteq V$ such that $|X| \leq k$ and $X$ is a \textit{vertex cover} of $G$, meaning that every edge in $E$ has at least one endpoint in $X$?}
\end{problem}

Observe that given a subset $X \subseteq V$, we can check in polynomial time if $|X| \leq k$ and if $X$ is a vertex cover of $G$. Thus, we can trivially answer the question in time $\O(2^n \cdot n^{\O(1)})$, where $n = |V|$, by testing every subset $X \subseteq V$. This complexity can be improved to $\O(\varphi^n \cdot n^{O(1)})$, where $\varphi \approx 1.618$ is the golden ratio, by observing two things: firstly, we can consider that every vertex has at least one neighbour, or we can remove it from the graph without changing the size of the smallest vertex cover; secondly, for every vertex $x$, if $x$ is not in a vertex cover $V$ of $G$, then every neighbour of $x$ needs to be in $V$. These observations lead to the following recurrence for the number of operations needed to compute the smallest vertex cover on a graph with $n$ vertices: $T(n) \leq T(n - 1) + T(n - 2)$; choose $x \in V$, and either include $x$ (removing $x$ from the graph), or exclude $x$ and include all its neighbours (removing $x$ and all its neighbours from the graph). This recurrence is similar to the Fibonacci sequence, which gives us the claimed time complexity.

\medskip

But can we refine our algorithm by introducing a parameter? Let us start with the integer $k$ given in the input. Observe that in the recursion tree given in the previous paragraph, we take at least one vertex at each step to include it in the vertex cover. Since the size of the vertex cover we are looking for cannot exceed $k$, the depth of the recursion tree should not exceed $k$, otherwise we can report a negative answer. Therefore, we have a $\O(2^k \cdot n^{\O(1)})$ algorithm for \prob{VertexCover}, parameterized by $k$. Note that this algorithm places \prob{VertexCover} parameterized by $k$ in \CC{FPT} with $f(k) = 2^k$.

While we will not go into the details of the techniques used, it is known that \prob{VertexCover} can be solved in time $\O(1.2738^k + n\cdot k)$ using kernelization and branching techniques \cite{chen2006improved}, which completely outperforms the $\O(\varphi^n \cdot n^{\O(1)})$ time algorithm since $k \leq n$.

\medskip

Let us explore another parameter: the \textit{treewidth} of the input graph $G$. We will define more precisely the treewidth of a graph in \refsec{section:preliminaries}, but for now, you may think of it as a measure of how similar the graph is to a tree. It is known that \prob{VertexCover} can be solved in time $\O(2^{\tw}\cdot\tw^{\O(1)}\cdot n)$, where $\tw$ denotes the treewidth of $G$, using dynamic programming techniques \cite[Corollary~7.6]{cygan2015parameterized}. Interestingly, it is also known that even though the dynamic programming techniques used in \cite[Corollary~7.6]{cygan2015parameterized} are among the most basic approaches in parameterized algorithms design, we cannot expect a better algorithm for \prob{VertexCover} parameterized by \tw{} unless the \textit{Strong Exponential Time Hypothesis} (SETH) fails \cite{lokshtanov2011known}. In other words, for every $\varepsilon > 0$, we do not expect any algorithm solving \prob{VertexCover} in time $\O((2-\varepsilon)^\tw \cdot n^{\O(1)})$ or it would imply that SETH does not hold. Recall that SETH is a conjecture which states that for every $\varepsilon > 0$, there exists no algorithm solving \prob{SAT} in time $\O((2 - \varepsilon)^n\cdot n^{\O(1)})$.

This lower bound on treewidth raises interesting questions: given that $\tw{} \leq n$ and there exists an algorithm with time complexity $\O(\varphi^n \cdot n^{\O(1)})$ but no algorithm with $\O((2 - \varepsilon)^\tw  \cdot n^{\O(1)})$, is there an intermediate parameter $p$ such that $\tw \leq p \leq n$ for which we can find an algorithm with time complexity $\O((2 - \varepsilon)^p \cdot n^{\O(1)})$ for any $\varepsilon > 0$?  Conversely, is there a parameter $p$ within the same range where it is impossible to achieve such an algorithm under SETH?

\medskip

Setting aside our example on \prob{VertexCover}, similar questions arise for many other \NP-hard problems. For several of these problems, we have a known dynamic programming approach parameterized by treewidth (see \reftheorem{theorem:treewidth-algo}). However, this approach is often tight, and under the SETH, we cannot expect a significantly faster algorithm (see \reftheorem{theorem:treewidth-bound}). Recall that the $\O^\star$ notation suppresses polynomial factors.

\begin{theorem}[\cite{lokshtanov2011known}]
    \label{theorem:treewidth-algo}
    Let $G$ be a graph, and let $\tw$ be the treewidth of $G$. It holds that:
    \begin{itemize}
        \item \prob{IndependentSet} can be solved in time $\O^\star(2^{\tw})$,
        \item \prob{DominatingSet} can be solved in time $\O^\star(3^{\tw})$,
        % \item \prob{MaxCut} can be solved in time $\O^\star(2^{\tw})$,
        \item \prob{OddCycleTransversal} can be solved in time $\O^\star(3^{\tw})$,
        \item \prob{$q$-Coloring} can be solved in time $\O^\star(q^{\tw})$,
        % \item \prob{PartitionIntoTrinagles} can be solved in time $\O^\star(2^{\tw})$.
    \end{itemize}
\end{theorem}

\begin{theorem}[\cite{lokshtanov2011known}]
    \label{theorem:treewidth-bound}
    Let $G$ be a graph, and let $\tw$ be the treewidth of $G$. If there exists an $\varepsilon > 0$ such that:
    \begin{itemize}
        \item \prob{IndependentSet} can be solved in time $\O^\star((2-\varepsilon)^{\tw})$, or
        \item \prob{DominatingSet} can be solved in time $\O^\star((3-\varepsilon)^{\tw})$, or
        % \item \prob{MaxCut} can be solved in time $\O^\star((2-\varepsilon)^{\tw})$, or
        \item \prob{OddCycleTransversal} can be solved in time $\O^\star((3-\varepsilon)^{\tw})$, or
        \item \prob{$q$-Coloring} can be solved in time $\O^\star((q-\varepsilon)^{\tw})$, or
        % \item \prob{PartitionIntoTrinagles} can be solved in time $\O^\star((2-\varepsilon)^{\tw})$.
    \end{itemize}
    then the SETH fails.
\end{theorem}

But for which parameters are these bounds still tight? Are the same parameters optimal for every problem? These questions were the focus during this internship, with the final goal being a refinement of our understanding of the lower and upper bounds for these problems.

In this report, we study several \NP-hard problems and establish a hierarchy of parameters. We focus on \textit{graph parameters}, such as the treewidth parameter discussed above. \refsec{section:preliminaries} delves into defintions and the knowledge we had before this internship. The subsequent sections focus on proving some new and interesting bounds.