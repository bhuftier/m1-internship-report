\subsection{Faster algorithm for $q$\prob{-Coloring}/$\shub$}
\label{section:qcol-shub}

This section presents an algorithm for \prob{$q$-Coloring}/$\shub$. The proof is just an extension of the proof for \prob{$q$-Coloring}/$\sdhub$ proposed by Esmer et al. \cite{esmer2024fundamental}.

\medskip

In the proof, we used the \prob{List-$q$-Coloring} problem, which is the following problem:

\begin{problem}
    \problemtitle{List-$q$-Coloring}
    \probleminput{A graph $G = (V, E)$, and a function $L : V \mapsto \pazocal{P}([q])$}
    \problemquestion{Can the vertices of $G$ be colored with $q$ colors such that no two adjacent vertices share the same color and each color respect the lists represented by $L$. In other words, does there exist a function $f : V \mapsto [q]$ where $f(u) \neq f(v)$ for every edge $\{u,v\} \in E$ and $f(u) \in L(u)$ for every vertex $u \in G$?}
\end{problem}

We can see that \prob{$q$-Coloring} is a specific case of \prob{List-$q$-Coloring} where all the lists are equal to $[q]$. Therefore, finding an algorithm for \prob{List-$q$-Coloring} instantly gives us an algorithm for \prob{$q$-Coloring}.

\medskip

\begin{theorem}
    For every $q \geq 3$ and every constant $\sigma \in \N$, there exists an $\varepsilon > 0$ with the follwing property: every instance $(G, L)$ of \prob{List-$q$-Coloring} with $n$ vertices, given with a $\shub$ of size $\ell$, can be solved in time $\O^\star((q-\varepsilon)^\ell)$.
\end{theorem}

\begin{proof}
    \todo{the first paragraphs of this proof are just copy-pasted from Esmer et al. paper (Section 7.1)}

    Let $X$ be a $\shub$ of $G$. If $X$ is a constant-size set, we can guess its coloring exhaustively, and then adjust the lists of the neighbours of vertices in $X$ as follows. If $x \in X$ is guessed to be colored with color $i$, then $i$ can be removed from lists of all vertices in the neighbourhood of $X$. After that we can safely remove $X$ from the graph, obtaining an equivalent instance, which can be solved in polynomial time as its every component is of constant size $\sigma$. Thus from now on assume that $|X| = \ell$ is sufficiently large.

    \medskip

    We first check if there is a vertex with an empty list; if so, we immediately reject the instance. Then we check whether there exists a component $A$ of $G - X$ with no neighbour in $X$. If such an $A$ exist, we solve the instance $G[A]$ of \prob{List-$q$-Coloring}: it can be done in constant time as $A \leq \sigma$. If it is a yes-instance, then we can proceed with the equivalent instance $G - A$; if it is a no-instance, then $(G, L)$ is a no-instance and we reject it.

    \medskip

    \todo{the new proof start here.}

    Fix a constant $\delta = q \times \sigma$. Suppose that for all components $A$ of $G - X$, we have $|N(A)| \leq \delta$, where $N(A)$ denotes the neighbourhood of $A$ in $X$. In this case, $X$ is also a $\sdhub$, and we can use the algorithm provided by Esmer et al. \cite{esmer2024fundamental} to solve the problem in time $\O^\star((q - \varepsilon)^\ell)$.

    \medskip

    Now, consider the case where there exists a component $A$ of $G - X$ such that $|N(A)| > \delta$. Since $|A| \leq \sigma$, this implies that there exists a vertex $u \in A$ such that $|N(u)| > q$. Let $U$ be a set consisting of $q$ elements from $N(u)$. It is impossible for the vertices in $U$ to be assigned $q$ distinct colors, as that would render $u$ uncolorable. Therefore, we can test all possible colorings of $U$ except those using exactly $q$ distinct colors, and then branch on $G - U$. This approach reduces the size of the hub by $q$ and involves $q^q - q!$ different branchings for $U$.

    The number of leaves in the recursion tree, measured as the function of $\ell$, is upper-bounded by the recursive formula
    $$T(\ell) \leq (q^q - q!) \cdot T(\ell - q)$$

    \todo{compute the real bound}
\end{proof}